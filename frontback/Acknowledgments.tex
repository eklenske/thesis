%*******************************************************
% Acknowledgments
%*******************************************************
% \pdfbookmark[1]{Acknowledgements}{acknowledgements}

\begin{flushright}{\slshape
The highest forms of understanding we can achieve\\
are laughter and human compassion.\\
--- Richard P. Feynman}
\end{flushright}

\begingroup
\let\clearpage\relax
\let\cleardoublepage\relax
\let\cleardoublepage\relax
\setlength{\parskip}{0.5\baselineskip}
\setlength{\parindent}{0pt}
\chapter*{Acknowledgements}

Even though there is only one author listed on the cover, I never was alone. I
was fortunate to receive constant support by many wonderful people during my PhD
adventures.

First of all, I want to thank all the past and present colleagues at the Max
Planck Institute for Intelligent Systems for the good times, for inspiring
discussions over coffee, lunch and foosball, and for the thought-provoking
caketalks and teatalks. I am grateful for having been surrounded by so many
smart people, from which I could easily learn something new every day.

Thanks to Sabrina, Andrea, Diana, Karin and Sebastian for running the
department so smoothly, and to Markus Schneller for tracking down old papers and
tech reports from the pre-digital age.

My work was partly supported by the Max Planck ETH Center for Learning Systems.
In addition to the financial support, I am grateful for the many opportunities
to travel to Z{\"u}rich and elsewhere to connect with the members and fellows
of the CLS.

There is a big difference between a piece of research code running in the lab,
and proper software working robustly on many devices. I want to thank Raffi
Enficiaud and the team of the Software Workshop at the Max Planck Institute in
T{\"u}bingen for their help in making it happen. Thanks to the developers of
PHD2 Guiding, especially Andy Galasso and Bruce Waddington, for the valuable
input and their help with including my work into the PHD2 Guiding project.

I want to thank Bernhard Sch{\"o}lkopf for giving me an amazing scientific home
for more than four years, for his expertise on telescope guiding,
and for the star gazing sessions in T{\"u}bingen, in the Black Forest and on La
Palma. It was an amazingly quiet and productive, but at the same time
exciting and inspiring working environment.

I owe my deepest gratitude to my supervisors Philipp Hennig and Melanie
Zeilinger, who invested a lot in my PhD. To Philipp, for the long hours in
front of the blackboard, for the PhD guidance and the constant support, and
for the invaluable advice, not limited to scientific matters. And to Melanie,
for teaching me so many things about control, and for always keeping a calm
mind despite the administrative obstacles. It was an honor working with you,
and I hope we can continue working together in the future.

Finally, I would like to thank Verena for her selfless support during the times
where being a PhD student was hard and exhausting, and for her love over all
these years.

\begin{flushright}
{\slshape Edgar Klenske}\\
Leinfelden, March 2017
\end{flushright}
\endgroup



