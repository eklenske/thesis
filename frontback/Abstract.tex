%*******************************************************
% Abstract
%*******************************************************
%\renewcommand{\abstractname}{Abstract}
\pdfbookmark[1]{Abstract}{Abstract}
\begingroup
% \let\clearpage\relax
\let\cleardoublepage\relax

\chapter*{Abstract}

Automatic control is an important aspect of modern technology, and many devices
we use on a daily basis are using automatic control for actuation and
decision-making. However, many advanced automatic control methods need a model
of the system to control---a mathematical representation of the system's
behavior. These models are not always easy to come by because of the underlying
complexity of the system or the required measurement precision. Therefore,
often a big portion of time is used for identification and tuning of these
models.

Machine learning methods with the available regression and inference
frameworks offer a new potential in the combination with model-based control
methods to speed up, or even entirely automate, the process of model-creation,
identification and tuning. This potential similarly extends to disturbance
prediction: Methods from time-series forecasting can be used to infer a
model of the environmental disturbances, which then can be used in predictive
control methods.

The first concept covered in this thesis is the identification of quasiperiodic
models for disturbance forecasting. Quasiperiodic disturbances are encountered
in many applications that are affected by the ubiquitous day-night-cycle,
revolving mechanical parts or recurring motions from biological objects.
Being able to forecast disturbances, such as the outside air temperature,
recurring gear errors or the beating motion of a heart, can help to increase
control performance of the affected systems, especially in combination with
model predictive control. In this thesis a quasiperiodic Gaussian process
regression framework is used to learn and to predict periodic disturbances. A
subsequent reference tracking model predictive controller then uses these
predictions to control the system to a higher precision. The benefits of using
this method are not only shown with simulated experiments, but also on
hardware, on a telescope tracking setup in the laboratory. Since the
development of this method was driven by a real problem in astronomical imaging,
it is described how the method was implemented as a software solution. The
advantage of the disturbance prediction method is shown on telescope tracking
experiments in the field. The use of automatically identified quasiperiodic
Gaussian process models makes it possible to use disturbance forecasting on a
variety of systems not necessarily known at modeling time.

The second concept presented in this thesis is nonlinear dual control. While
methods for simultaneous identification and control were published already in
the 1960s, most approaches are either too complex to be used in practice or too
simple to retain all critical features of the original dual control framework:
\emph{caution}, \emph{exploration} and \emph{selectiveness}. The dual control
framework in this thesis is based on one approximation to the---theoretically
ideal, but fundamentally intractable---optimal dual control problem. So far
being used for linear systems only, this framework is extended to nonlinear
systems. This is done by employing regression methods from machine learning;
namely parametric regression, Gaussian process regression and neural network
regression. Furthermore, the framework, which was originally only suitable for
systems with quadratic cost, is extended to a general cost setting. This makes
it possible to apply dual control to problems of economic cost. An exemplary
application to a nonlinear building control problem shows the potential that
dual control offers for real world applications. Overall, the presented
extensions make it possible to use approximation-based dual control in the
context of nonlinear regression models and flexible cost structures.

\clearpage

\begin{otherlanguage}{ngerman}
\pdfbookmark[1]{Zusammenfassung}{Zusammenfassung}
\chapter*{Zusammenfassung}

Regelungstechnik ist ein wichtiger Bestandteil der modernen Technik, und viele
Geräte, die wir täglich nutzen, arbeiten nur mit Hilfe von Reglern zuverlässig.
Viele moderne Regler benötigen dynamische Modelle---mathematische
Beschreibungen des Systemverhaltens---um zu funktionieren. Leider ist es
aufgrund der Komplexität und der erforderlichen Messgenauigkeit nicht immer
einfach, diese Modelle zu erstellen. Deshalb ist die
Sys\-tem\-iden\-ti\-fi\-ka\-ti\-on und das Anpassen der Modelle an die realen
Begebenheiten oftmals mit einem hohen Zeiteinsatz verbunden.

Methoden des maschinellen Lernens bieten die Möglichkeit, den Prozess der
Modellierung und Anpassung zu beschleunigen oder gänzlich zu automatisieren.
Dieses Potential gilt auch für Methoden zur Vorhersage von Störungen:
Algorithmen zur Zeitreihenvorhersage können genutzt werden, um den zukünftigen
Verlauf externer Störungen vorherzusagen. Diese Vorhersagen können anschließend
in der modellprädiktiven Regelung genutzt werden, um die Regelgenauigkeit zu
verbessern.

Das erste Konzept, welches in dieser Arbeit behandelt wird, ist die
Identifikation von qua\-si\-pe\-ri\-o\-disch\-en Modellen für die Vorhersage
von Störungen. Quasiperiodische Störungen findet man in vielen Anwendungen,
insbesondere bei solchen, die vom Tag- und Nacht-Zyklus, rotierenden
mechanischen Teilen oder anderen periodischen Bewegungen abhängen. Die
Möglichkeit, Störungen wie die Außentemperatur, Getriebefehler oder das
Schlagen eines Herzens vorherzusagen, kann die Regelgenauigkeit der betroffenen
Systeme erheblich verbessern. In dieser Arbeit wird ein quasiperiodischer
Gauß-Prozess eingesetzt, um periodische Störungen zu modellieren und
vorherzusagen. Ein Referenzfolgesystem auf Basis modellprädiktiver Regelung
nutzt diese Vorhersagen, um die Nachführgenauigkeit des Reglers zu verbessern.
Die Vorteile dieser Methode werden nicht nur mit simulierten Experimenten,
sondern auch auf einem mechanischen Versuchsträger, einem Teleskop-Aufbau im
Labor, gezeigt. Für die Verbesserung der Regelung von Teleskopen in der
Astrophotographie wurde der Algorithmus in einer Softwarelösung implementiert
und der Nutzen der Störungsvorhersage mit Experimenten im Feldversuch
demonstriert. Mit dieser Softwarelösung ist es nun möglich, die
Störungsvorhersage auf einer Vielzahl von Systemen zu nutzen, die zur Zeit der
Modellierung nicht notwendigerweise bekannt sein müssen.

Das zweite Konzept in dieser Arbeit ist die nichtlineare duale Regelung.
Während Methoden für die zeitgleiche Identifikation und Regelung dynamischer
Systeme schon in den 1960er-Jahren publiziert wurden, sind die meisten Ansätze
entweder zu komplex um in der Praxis eingesetzt zu werden, oder zu einfach um
alle wichtigen Eigenschaften der dualen Regelung zu erhalten: \emph{Vorsicht},
\emph{Exploration} und \emph{Selektivität}. Der Ansatz in dieser Arbeit basiert
auf einer Approximation des---theoretisch idealen, aber praktisch nicht
lösbaren---optimalen dualen Regelungsproblems. Diese bisher nur für lineare
Systeme eingesetzte Methode wird auf nichtlineare Modelle erweitert. Dies wird
durch die Nutzung verschiedener nichtlinearer Regressionsmethoden erreicht:
parametrische Regression, Gauß-Prozess-Regression und Regression auf Basis
neuronaler Netze. Weiter wurde der Ansatz, der bisher nur für Systeme mit
quadratischen Kosten nutzbar war, auf allgemeine Kostenstrukturen erweitert.
Dies ermöglicht es, duale Regelung auch auf Probleme mit ökonomischen Kosten
anzuwenden. Eine beispielhafte Anwendung auf nichtlineare Gebäuderegelung zeigt
das Potential, welches duale Regelung für praktische Anwendungen bietet.
Insgesamt machen es die vorgestellten Erweiterungen möglich, approximative duale
Regelung im Kontext nichtlinearer Regressionsmodelle und flexiblen
Kostenstrukturen einzusetzen.

\end{otherlanguage}

\endgroup
